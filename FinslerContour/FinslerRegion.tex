\documentclass[11pt]{article}
\usepackage[utf8]{inputenc}
\usepackage[T1]{fontenc}
%\usepackage[french]{babel}

\usepackage{enumerate}
\usepackage{url}
\usepackage{epsf}
\usepackage{epsfig}
\usepackage{graphicx}
\usepackage{verbatim}


\voffset=-2cm
%\voffset=-3cm
\hoffset=-1.5cm
\textwidth=16cm
\textheight=23cm

\usepackage{amsmath,amsfonts,amssymb,amsthm}
\usepackage{wasysym}
\providecommand*\showkeyslabelformat[1]{\hspace{-4mm}\underline{\normalfont\small\ttfamily#1}}
\usepackage{stmaryrd}

\usepackage{ gensymb }
\usepackage{ esint }

\usepackage{hyperref}
\usepackage[notref]{showkeys}


\usepackage{algorithm,algpseudocode,algorithmicx}
%\usepackage{algpseudocode}
%\usepackage{algorithm}

\newtheorem{definition}{Definition}
\newtheorem{remark}[definition]{Remark}
\newtheorem{lemma}[definition]{Lemma}

%\newenvironment{obviousProof}{\begin{proof}[Obvious]}{\end{proof}}
\let\obviousProof\comment   \let\endobviousProof\endcomment

\def\bR{{\mathbb R}}
\def\mR{{\mathbb R}}
\def\cE{{\mathcal E}}
\DeclareMathOperator\rot{rot}
\DeclareMathOperator\length{length}
\def\<{\langle} \def\>{\rangle}

\author{
Jean-Marie Mirebeau
}
\begin{document}
\title{
Minimizing Chan-Vese type energies\\ using Finsler minimal paths.
}
\maketitle
\date{}

\section{Introduction}

To each domain $\Omega\subset \mR^2$ attach the energy
\begin{equation}
\label{eqdef:EOmega}
	\cE(\Omega) := \int_\Omega \rho+\int_{\partial \Omega} \sigma,
\end{equation}
where $\rho : \mR^2 \to \mR$ has arbitrary sign, and $\sigma : \mR^2 \to \mR_+^*$ is strictly positive. If $\rho$ has negative sign, then the integral over $\Omega$ acts as a balloon force.
The objective is to (locally) minimize $\cE$ among domains $\Omega \subset \mR^2$.

\begin{definition}
	Let $\Omega\subset \mR^2$ be open, simply connected, with Lipschitz boundary, and let $x_0,x_1 \in \partial \Omega$. Then one can parametrize $\partial \Omega$ as the union of two Lipschitz curves $\gamma_+,\gamma_-$, with common endpoints $\gamma_+(0)=\gamma_-(0)=x_0$ and $\gamma_+(1)=\gamma_-(1)=x_1$, going respectively counter-clockwise and clockwise.
	We write
	\begin{equation*}
		\Omega = [\gamma_+, \gamma_-].
	\end{equation*}
	Conversely, let $\gamma_+,\gamma_-$ be two curves which are Lipschitz, without  self intersections, and which do not intersect each other except at their common endpoints. Then perhaps exchanging the roles of $\gamma_+$ and $\gamma_-$ for orientation one has $\Omega = [\gamma_+, \gamma_-]$.
	\end{definition}


Let $\Omega = [\gamma_+, \gamma_-]$, and let $x_0=\gamma_+(0)=\gamma_-(0)$, $x_1=\gamma_+(1) = \gamma_-(1)$.
We design neighborhoods $V_+, V_-$ of $\gamma_+, \gamma_-$, path lengths $\length_+, \length_-$ with respect to adequate Finsler metrics on $V+,V_-$, and a potential function $p : V_+\cap V_- \to \mR$.
Then for any curves $\Gamma_+, \Gamma_-$ within $V_+,V_-$, originating from $x_0=\Gamma_+(0)=\Gamma_-(0)$, and with common endpoint $X_1$ one has 
\begin{equation*}
	\cE([\Gamma_+, \Gamma_-]) = \length_+(\Gamma_+) + \length_-(\Gamma_-) - p(X_1)+p(x_0).
\end{equation*}
Denote by $d_+, d_-$ the geodesic distance from $x_0$ with respect to $\length_+, \length_-$. Then for improving \eqref{eqdef:EOmega} it suffices to minimize
\begin{equation*}
	d_+(X_1)+d_-(X_1) - p(X_1)
\end{equation*}
and then extract shortest paths $\Gamma_+, \Gamma_-$. The paths are then reversed and exchanged $[-\Gamma_-, -\Gamma_+] = [\Gamma_+, \Gamma_-]$, in order to exchange the roles of $x_0,X_1$, and the procedure is repeated.

\section{Defining the metrics}


\begin{lemma} (*Which are the correct spaces ? What estimates on the growth of $\Omega$ ?*)
	Let $V \subset \bR^2$ be a bounded domain, let $\gamma : [0,1] \to \bR^2$ be a Lipschitz curve without self-intersections, and let $\rho : V \to \bR^2$.
	Then there exists $\omega : V \to \bR^2$ such that (*?? Probably not, but what is the right estimate ??*) $|\omega(x)| \leq \|\rho\|_\infty d(x,\gamma)$.
\end{lemma}

Let $V_+$ (resp.\ $V_-$) be a neighborhood of $\gamma_+$ (resp.\ $\gamma_-$), and $\omega_+ : V_+ \to \mR^2$ (resp.\ $\omega_-$) a solution to
\begin{align*}
	\rot \omega_+ &= \rho,
	& \rot \omega_- &= - \rho.
\end{align*}
We choose these vector fields 

We denoted $\rot \omega := \partial_1 \omega_2-\partial_2\omega_1$.
By construction one has $\rot(\omega_+ + \omega_-) = 0$ on $V_+ \cap V_-$, hence there exists $p : V_+ \cap V_- \to \mR$ such that 
\begin{equation*}
	\omega_+ +\omega_- = \nabla p.
\end{equation*}
The solution $\omega_+$ to $\rot \omega_+ = \rho$ can be extended globally on $\mR^2$, hence denoting by $\tau$ the counter-clockwise tangent vector to $\partial \Omega$ we can write
\begin{align*}
	\int_\Omega \rho &= \int_\Omega \rot \omega_+\\
	&= \int_{\partial \Omega} \<\omega_+,\tau\>\\
	&= - \int_0^1 \<\omega_+,\gamma_+'\> + \int_0^1 \<\omega_+,\gamma_-'\>\\
	&= - \int_0^1 \<\omega_+,\gamma_+'\> + \int_0^1 \<\nabla p - \omega_-,\gamma_-'\>\\
	&= - \int_0^1 \<\omega_+,\gamma_+'\> - \int_0^1 \<\omega_-,\gamma_-'\> + p(y)-p(x).
\end{align*}

For any path 
(*...*)

\begin{equation*}
	\cE([\gamma_+,\gamma_-]) = \length_+(\gamma_+) + \length_-(\gamma_-).
\end{equation*}

\begin{remark}[Divergence theorem]
Denoting by $R$ the counter-clockwise rotation by $\pi/2$, one has $\rot \omega = \nabla\cdot R \omega$ and $\tau = R n$ where $n$ is the exterior normal.
Thus using the identity $R^T = -R$
\begin{align*}
	\int_\Omega \rot \omega = \int_\Omega \nabla \cdot R \omega = \int_{\partial \Omega} \<R \omega , n\> = \int_{\partial \Omega} \<\omega, R^T n\> = -\int_{\partial \Omega} \<\omega,\tau\>.
\end{align*}
\end{remark}

\section{Implementation}

%\bibliographystyle{alpha}
%\bibliography{../../AllPapers}

\end{document}
